\documentclass[letterpaper, 10 pt, conference]{ieeeconf}  % Comment this line out if you need a4paper

%\documentclass[a4paper, 10pt, conference]{ieeeconf}      % Use this line for a4 paper

\IEEEoverridecommandlockouts                              % This command is only needed if 
                                                          % you want to use the \thanks command

\overrideIEEEmargins                                      % Needed to meet printer requirements.

%\renewcommand{\baselinestretch}{2}

% See the \addtolength command later in the file to balance the column lengths
% on the last page of the document
\pdfminorversion=4
\input{macros}

\title{\LARGE \bf Synthesis of Surveillance Strategies via Belief Abstraction}

\author{Suda Bharadwaj$^{1}$ and Rayna Dimitrova$^{1}$ and Ufuk Topcu$^{1}$% <-this % stops a space
%\thanks{*This work was not supported by any organization}% <-this % stops a space
\thanks{$^{1}$Suda Bharadwaj, Rayna Dimitrova and Ufuk Topcu are with the University of Texas at Austin}%
%\thanks{$^{2}$Bernard D. Researcheris with the Department of Electrical Engineering, Wright State University, Dayton, OH 45435, USA {\tt\small b.d.researcher@ieee.org}}%
}



\begin{document}

\maketitle
\thispagestyle{empty}
\pagestyle{empty}

%%%%%%%%%%%%%%%%%%%%%%%%%%%%%%%%%%%%%%%%%%%%%%%%%%%%%%%%%%%%%%%%%%%%%%%%%%%%%%%%

\begin{abstract}
<<<<<<< HEAD
We study the problem of synthesizing a controller for a robot with a \emph{surveillance objective}, that is, the robot is required to  maintain knowledge of the location of a moving, possibly adversarial target. We formulate this problem as a one-sided partial-information  game in which the winning condition for the agent is specified as a temporal logic formula. The specification formalizes the surveillance requirement given by the user, including additional non-surveillance tasks. In order to synthesize a surveillance strategy that meets the specification, we transform the partial-information game into a perfect-information one, using abstraction to mitigate the exponential blow-up typically incurred by such transformations. This transformation enables the use of off-the-shelf tools for reactive synthesis. We use counterexample-guided refinement to automatically achieve abstraction precision that is sufficient to synthesize a surveillance strategy. We evaluate the proposed method on two case-studies, demonstrating its applicability to large state-spaces and diverse requirements.\looseness=-1
=======
%<<<<<<< HEAD
%We study the problem of synthesizing a controller for a robot with a \emph{surveillance objective}, that is, the robot is required to  maintain knowledge of the location of a moving, possibly adversarial target. We formulate this problem as a one-sided partial-information  game in which the winning condition for the agent is specified as a temporal logic formula. The specification formalizes the surveillance requirement given by the user, including additional non-surveillance tasks. In order to synthesize a surveillance strategy that meets the specification, we transform the partial-information game into a perfect-information one, using abstraction to mitigate the exponential blow-up typically incurred by such transformations. This enables the use of off-the-shelf tools for reactive synthesis. 
%%We use counterexample-guided refinement to automatically achieve abstraction precision that is sufficient to synthesize a surveillance strategy.
%We evaluate the proposed method on two case-studies, demonstrating its applicability to large state-spaces and diverse requirements.\looseness=-1
%=======

We study the problem of synthesizing a controller for a robot with a \emph{surveillance objective}, that is, the robot is required to  maintain knowledge of the location of a moving, possibly adversarial target. We formulate this problem as a one-sided partial-information  game as the location of the target is not always known. We transform the partial-information game into a perfect-information one, using abstraction to mitigate the exponential blow-up typically incurred by such transformations. We then leverage this reduction by assigning \emph{surveillance predicates} to the belief states allowing us to specify the winning condition for the agent as a temporal logic formula. The specification formalizes the surveillance requirement given by the user, including additional non-surveillance tasks. This framework enables the use of off-the-shelf tools for reactive synthesis. We demonstrate the ability of this framework in experiments to tailor different surveillance behaviour characteristics based on the needs of the user.

\looseness=-1


%We study the problem of synthesizing a controller for a robot with a \emph{surveillance objective}, that is, the robot is required to  maintain knowledge of the location of a moving, possibly adversarial target. We formulate this problem as a one-sided partial-information  game in which the winning condition for the agent is specified as a temporal logic formula. The specification formalizes the surveillance requirement given by the user, including additional non-surveillance tasks. In order to synthesize a surveillance strategy that meets the specification, we transform the partial-information game into a perfect-information one, using abstraction to mitigate the exponential blow-up typically incurred by such transformations. This enables the use of off-the-shelf tools for reactive synthesis. We use counterexample-guided refinement to automatically achieve abstraction precision that is sufficient to synthesize a surveillance strategy. We evaluate the proposed method on two case-studies, demonstrating its applicability to large state-spaces and diverse requirements.\looseness=-1
%>>>>>>> faa5d5fb9b1a257dc7138f42a89542f0a4e8a78e
>>>>>>> f91acc1710941992159fff02eb617b87fdff8ddc




%given a linear temporal logic specification, as well as a surveillance objective where it is required to maintain knowledge of the location of a moving, possibly adversarial target. We formulate this problem as a one-sided partial-information game. We then reduce the partial-information game to a full-observation game using an abstracted belief set construction to avoid the exponential blow up of the state space. In order to ensure the surveillance requirement is satisfied, we allow it to be encoded in three different ways : a safety objective, a liveness objective, or both depending on the qualitative behavioural requirement for the user. This requirement is then appended to the original specification and formulated as an assume-guarantee problem for reactive synthesis on the abstract belief-set game. We then use a counterexample-guided abstraction refinement scheme to refine the abstract belief states until a reactive controller can be synthesized.
\end{abstract}

%%%%%%%%%%%%%%%%%%%%%%%%%%%%%%%%%%%%%%%%%%%%%%%%%%%%%%%%%%%%%%%%%%%%%%%%%%%%%%%%

\section{INTRODUCTION}
\input{intro}


%%%%%%%%%%%%%%%%%%%%%%%%%%%%%%%%%%%%%%%%%%%%%%%%%%%%%%%%%%%%%%%%%%%%%%%%%%%%%%%%

\section{GAMES WITH SURVEILLANCE OBJECTIVES}
\input{games}

%%%%%%%%%%%%%%%%%%%%%%%%%%%%%%%%%%%%%%%%%%%%%%%%%%%%%%%%%%%%%%%%%%%%%%%%%%%%%%%%

\section{BELIEF SET ABSTRACTION}
%\subsection{Abstract Belief-Set Games}
\input{abstraction}

%\subsection{Soundness of Belief Set Abstraction}
In the construction of the abstract  game structure, we overapproximate the belief-set of the agent at each step. Since we consider surveillance predicates that impose upper bounds on the size of the belief, such an abstraction  gives more power to the target (and, dually less power to the agent).  This construction guarantees that the abstraction is \emph{sound}, meaning that an abstract strategy for the agent that achieves a surveillance objective corresponds to a winning strategy in the concrete game. This is stated in the following theorem.

\begin{theorem}
Let $G$ be a surveillance game structure, $\part = \{Q_i\}_{i=1}^n$ be an abstraction partition, and $G_\abstr = \alpha_\part(G)$. For every surveillance objective $\varphi$, if there exists a wining strategy for the agent in the abstract belief-set game $(\alpha_\part(G),\varphi)$, then there exists a winning strategy for the agent in the concrete surveillance game $(G,\varphi)$.
\end{theorem}

\paragraph{Choosing an abstraction partition} Choosing an 'appropriate' partition can result in a controller that can guarantee the surveillance specifications with fewer abstract belief states. However, choosing the abstraction partition is, in general, specific to the game environment. In section \ref{sec:experiments}, we present some examples with user specified partitions. If the abstraction is too coarse, the user can refine the abstract states with the help of counterexamples. The nature of these counterexamples for the classes of surveillance specifications presented in this papers are defined in the next section.

The refinement procedure can be automated (CEGAR), however, for space constraints we do not include the details in this paper. 


%%%%%%%%%%%%%%%%%%%%%%%%%%%%%%%%%%%%%%%%%%%%%%%%%%%%%%%%%%%%%%%%%%%%%%%%%%%%%%%%

\section{ ABSTRACTION PRECISION}
%\section{COUNTEREXAMPLES}% SURVEILLANCE OBJECTIVES}
The ideal choice of an abstraction partition is the one that balances precision and computational burden. More precisely, the abstraction should be precise enough for the agent to satisfy its surveillance objective. On the other hand, an abstraction that is too precise, often results in an intractably large state-space of the resulting game. Thus, a good abstraction is one that gives the right level of precision where it is needed, and is coarse (that is, generates fewer abstract belief states) where precision is not needed. Thus, choosing a good abstraction partition is often specific to the game environment and the surveillance specification. In section \ref{sec:experiments}, we present  examples with user specified partitions resulting in feasible  abstract games.

In the previous section, we discussed that a winning strategy for the agent in the abstract belief game corresponds to a strategy for the agent in the concrete belief game. This, fact does not hold in general for the abstract winning strategies of the target. We refer to the abstract winning  strategies for the target as \emph{abstract counterexamples}.

Given an abstract counterexample, there are two possibilities: it can either be a counterexample in the concrete belief game, meaning that the agent cannot satisfy the surveillance objective, or it may exist due to the coarseness of the abstraction partition. We now discuss in more detail counterexamples in safety and liveness surveillance games. The latter generalizes also to general surveillance objectives.

\subsection{Counterexamples  for Safety Surveillance Properties}
\input{counterex-safety}
%\subsection{Counterexample-Guided Refinement}
%\input{refinement-safety}
%
%%%%%%%%%%%%%%%%%%%%%%%%%%%%%%%%%%%%%%%%%%%%%%%%%%%%%%%%%%%%%%%%%%%%%%%%%%%%%%%%%
%
%\section{BELIEF REFINEMENT FOR LIVENESS}% SURVEILLANCE OBJECTIVES}
\subsection{Counterexamples for Liveness Surveillance Properties}
\input{counterex-liveness}
%\subsection{Counterexample-Guided Refinement}
%\input{refinement-liveness}
%\subsection{General surveillance and task specifications}
%\input{general-objectives}
\subsection{Counterexample-Guided Refinement}
Since the game structures we consider are finite, and counterexamples have finite representation, we can effectively determine whether an abstract counterexample is concretizable or not. In the first case we report the unrealizability of the surveillance objective. In the second case, the counterexample can be used to determine which parts of the abstraction need to be refined in order to eliminate this, and possibly other, counterexamples. This analysis and refinement procedure can be automated, and we refer the reader to \cite{arxiv} for more details on the process. We remark that when the refinement is automated, the choice of the initial abstraction often plays a crucial role in keeping the size of the abstract game within limits that make the application of synthesis tools feasible.

%%%%%%%%%%%%%%%%%%%%%%%%%%%%%%%%%%%%%%%%%%%%%%%%%%%%%%%%%%%%%%%%%%%%%%%%%%%%%%%%

\section{EXPERIMENTAL EVALUATION}\label{sec:experiments}
We report on the application of our method for surveillance synthesis to two case studies. We have implemented the simulation in \texttt{Python}, using the \texttt{slugs} reactive synthesis tool~\cite{EhlersR16}. The experiments were performed on an Intel i5-5300U 2.30 GHz CPU with 8 GB of RAM. 

\subsection{Liveness surveillance specification + task specification}
Figure~\ref{fig:case1} shows a gridworld divided into  'rooms'. The surveillance objective requires the agent to infinitely often know precisely the location of the target (either see it, or have a belief consisting of one cell). Additionally, it has to perform the task of patrolling (visiting infinitely often) the green 'goal' cell. Formally, the specification is $\LTLglobally\LTLfinally p_1 \wedge \LTLglobally\LTLfinally \mathit{goal}$. The agent can move between $1$ and $3$ cells at a time. The sensor model used here is 'line-of-sight' with a range of 5 cells. The agent cannot see through obstacles (shown in red) and cannot see further than 5 cells. 


\begin{figure}
\subfloat[Gridworld with a user provided abstraction with 7 abstract belief states marked in black lines. \label{fig:case1}]{
\includegraphics[scale=0.3]{figs/Liveness_part.png}
}
\hfill
\subfloat[Gridworld showing visibility of the agent. All states in black are invisible to the agent. \label{fig:case1vis}]{
\includegraphics[scale=0.3]{figs/Liveness_t1.png}\hspace{.5cm}
}

\caption{10x15 gridworld with a surveillance liveness specification. The agent is blue, and the target to be surveilled is orange. Red states are obstacles.}
\label{fig:casestudies}

\end{figure}



Using the 7 abstract states as shown in Figure \ref{fig:case1}, the overall number of states in the two-player game is $15\times10 + 2^7 = 278$ states. In contrast, solving the full abstract game will have in the order of $2^{150}$ states, which is a state-space size that state-of-the-art synthesis tools cannot handle. 

\begin{figure}
\begin{minipage}{5.0cm}
	\centering
		\subfloat[$t_1$ \label{fig:case1t2}]{
		\includegraphics[scale=0.17]{figs/Liveness_t2.png}\hspace{.5cm}
	}
	\subfloat[$t_3$ \label{fig:case1t3}]{
		\includegraphics[scale=0.17]{figs/Liveness_t3.png}\hspace{.5cm}
	}
	\subfloat[$t_4$ \label{fig:case1t4}]{
	\includegraphics[scale=0.17]{figs/Liveness_t4.png}\hspace{.5cm}
}
\end{minipage}
\begin{minipage}{5.0cm}
	\centering
	\subfloat[$t_5$  \label{fig:case1t5}]{
		\includegraphics[scale=0.17]{figs/Liveness_t5.png}\hspace{.5cm}
	}
	\subfloat[$t_6$ \label{fig:case1t6}]{
		\includegraphics[scale=0.17]{figs/Liveness_t6.png}\hspace{.5cm}
	}
	\subfloat[$t_7$ \label{fig:case1t7}]{
		\includegraphics[scale=0.17]{figs/Liveness_t7.png}\hspace{.5cm}
	}
	
\end{minipage}

	
	\caption{Evolution of the agent's belief in target location as it moves to the goal and loses sight of the target. Grey states represents the states the agent believes the target could be in. We show the belief at different timesteps (note that $t_2$ is excluded for space concerns)
		}
	\label{fig:case1exp}
	
\end{figure}

Figure \ref{fig:case1exp} shows how when the agent cannot see the target, the belief (shown in grey) can grow quickly. This growth occurs due to the coarseness of the abstraction, which overapproximates the target's true position. In 7 timesteps, the agent believes the target can exist anywhere in the grid that is not in its vision. It has to then find the target in order to satisfy the surveillance requirement. In this example, 7 abstract states were enough to guarantee the satisfaction surveillance specification, but for comparison, we also solve the game with 12 abstract states to illustrate the change in growth in belief. Figure \ref{fig:case1fineexp} shows the belief states growing much more slowly as the abstract belief states are smaller so it more closely approximates the true belief.

\begin{figure}
	\begin{minipage}{5.0cm}
		\centering
		\subfloat[$t_1$ \label{fig:casefine1t2}]{
			\includegraphics[scale=0.17]{figs/Liveness_t2.png}\hspace{.5cm}
		}
		\subfloat[$t_3$ \label{fig:case1finet3}]{
			\includegraphics[scale=0.17]{figs/Liveness_t3.png}\hspace{.5cm}
		}
		\subfloat[$t_4$ \label{fig:case1finet4}]{
			\includegraphics[scale=0.17]{figs/Liveness_t4.png}\hspace{.5cm}
		}
	\end{minipage}
	\begin{minipage}{5.0cm}
		\centering
		\subfloat[$t_5$  \label{fig:case1finet5}]{
			\includegraphics[scale=0.17]{figs/Liveness_t5.png}\hspace{.5cm}
		}
		\subfloat[$t_6$ \label{fig:case1finet6}]{
			\includegraphics[scale=0.17]{figs/Liveness_t6.png}\hspace{.5cm}
		}
		\subfloat[$t_7$ \label{fig:case1finet7}]{
			\includegraphics[scale=0.17]{figs/Liveness_t7.png}\hspace{.5cm}
		}
		
	\end{minipage}
	
	
	\caption{Evolution of the agent's belief in target location when the game is solved with 12 abstract belief states.
	}
	\label{fig:case1fineexp}
	
\end{figure}

%Starting with an abstract game with 104 states generated by a partition with four elements, our refinement algorithm terminates after 5 iterations (with total running time of 821 s). The resulting partition $\mathcal{Q} = \{Q_1,...,Q_9 \}$ has $9$ elements shown as the numbered regions in Figure~\ref{fig:case1}. Thus, the final refined abstract game has $616$ abstract states ($2^9$ abstract belief states). In contrast, the belief-set game structure would have in the order of $2^{100}$ states, which is a state-space size that state-of-the-art synthesis tools cannot handle.


%The refinement algorithm terminates after 5 iterations to produce the abstract partition  corresponding to the numbered sets in figure . There are 9 belief states which results in an additional $2^9 = 512$ states to the full observation game which is far lower than the full reduction which will be a power set of all 104 states.

The additional abstract belief states results in a much larger game as the state space grows exponentially in the number of abstract belief states. Table \ref{tab:exp1} compares the state spaces and the amount of time it takes to synthesize a controller.


\begin{table}[h!]
	\centering
	\begin{tabular}{c|c|c}
		Abstract states & Total states & Synthesis time \\ \hline \hline
		7 & 278 & 37s \\ 
		12 & 4346 & 527s \\ 
	\end{tabular}\caption{Comparison of synthesis times for the two cases} \label{tab:exp1}
\end{table}


A video simulation can be found at \url{http://goo.gl/YkFuxr}. Note the behaviour of the agent - visiting the goal and then searching for the target. This contrasts with the behaviour under safety surveillance which will we look at next.

\subsection{Safety surveillance specification + task specification}
%\begin{figure}
%\centering
%\includegraphics[scale=0.2]{case2.png}\caption{Gridworld representing an outdoor environment.}\label{fig:case2}
%\vspace{-.5cm}
%\end{figure}
Figure~\ref{fig:case2} depicts a gridworld of an 'outdoor' environment where the red blocks model buildings. 
In this setting, we enforce the safety surveillance objective $\square p_{30}$ (the belief size should never exceed 30) in addition to infinitely often reaching the green cell. The formal specification is $\LTLglobally\LTLfinally p_{30} \wedge \LTLglobally\LTLfinally \mathit{goal}$. Additionally, the target itself is trying to reach the goal cell infinitely often as well, which is known to the agent.

We used an abstraction generated by a partition of size 6, which was sufficiently precise to compute a surveillance strategy in 210 s. This demonstrates that even for larger grids, a coarse abstraction can be sufficient. Again, note that the precise belief-set game would have in the order of $2^{200}$ states.
 
We simulated the environment and the synthesized surveillance strategy for the agent in ROS. A video of the simulation can be found at \url{http://goo.gl/LyC1gQ}. Note the qualitative difference in behaviour compared to the previous example. There, in the case of liveness surveillance, the agent had more leeway to completely lose the target in order to reach its goal location, even though the requirement of reducing the size of the belief to $1$ is quite strict. Here, on the other hand, the safety surveillance objective, even with a large threshold of $30$, forces the agent to follow the target more closely, in order to prevent its belief from getting too large. The synthesis algorithm thus provides the ability to obtain qualitatively different behaviour as necessary for specific applications by combining different objectives. 

%\Suda{Not sure if we should keep this next part}
%\subsection{Discussion of behaviour}
%The difference in the behaviour in the case studies highlights the different use cases of the surveillance objectives. In more indoor settings or structured environments, a liveness surveillance objective is feasible as the agent can more easily search and find the target even if the belief grows very large. However, in outdoor environments this is harder to accomplish as the target has more room to hide. 

%%%%%%%%%%%%%%%%%%%%%%%%%%%%%%%%%%%%%%%%%%%%%%%%%%%%%%%%%%%%%%%%%%%%%%%%%%%%%%%%

\section{CONCLUSIONS}
We have presented a novel approach to solving a surveillance problem with information guarantees. We provide a framework to set up the surveillance problem as a two-player, partial-information game. We then present a method to reason over the belief that the agent has over the target's location and specify formal surveillance requirements. The user can tailor the behaviour to their specific application by using a combination of safety and liveness surveillance objectives.

The benefit of this framework is that techniques commonly used in reactive synthesis can be incorporated in this problem. Among others, two avenues of future work in this framework we are currently looking into include: 
\begin{itemize}
\item Using CEGAR to automate the choice and refinement of the abstraction partition.
\item Use compositional synthesis to synthesize decentralized surveillance strategies for multiple agents at once. This will avoid the blow up of the state space that will occur if the multiple agents are solved in a centralized manner.
\end{itemize}

%%%%%%%%%%%%%%%%%%%%%%%%%%%%%%%%%%%%%%%%%%%%%%%%%%%%%%%%%%%%%%%%%%%%%%%%%%%%%%%%

%\section*{APPENDIX}

%%%%%%%%%%%%%%%%%%%%%%%%%%%%%%%%%%%%%%%%%%%%%%%%%%%%%%%%%%%%%%%%%%%%%%%%%%%%%%%%

%\section*{ACKNOWLEDGMENT}


%%%%%%%%%%%%%%%%%%%%%%%%%%%%%%%%%%%%%%%%%%%%%%%%%%%%%%%%%%%%%%%%%%%%%%%%%%%%%%%%
\bibliographystyle{IEEEtran}
\bibliography{main}

\end{document}
