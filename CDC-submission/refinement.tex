Since the game structures we consider are finite, and counterexamples have finite representation, we can effectively determine whether an abstract counterexample is concretizable or not. In the first case we report the unrealizability of the surveillance objective. In the second case, the counterexample can be used to determine which parts of the abstraction need to be refined in order to eliminate this, and possibly other, counterexamples. This analysis and refinement procedure can be automated, and we refer the reader to \textbf{ARXIV} for more details on the process. We remark that when the refinement is automated, the choice of the initial abstraction often plays a crucial role in keeping the size of the abstract game within limits that make the application of synthesis tools feasible.
