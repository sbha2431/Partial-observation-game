A winning strategy for the target in a game with safety surveillance objective can be naturally represented as a tree. 
An \emph{abstract counterexample tree} $\counterex_\abstr$ for $(G_\abstr,\LTLglobally p_k)$ is a finite tree,  whose nodes are labelled with states in $\states_\abstr$ such that the following conditions are satisfied:
\begin{itemize}
\item The root node is labelled with the initial state $s_\abstr^\init$;
\item A node is labelled with an abstract state  which violates $p_k$ (that is, $s_\abstr$ where $s_\abstr \not\models p_k$) iff it is a leaf;
\item The tree branches according to all possible transition choices of the agent. Formally, if an internal node $v$ is labelled with $(l_a,A_t)$, then there exists an $A_t'$  such that: (1) $((l_a,A_t),(l_a',A_t')) \in \trans_\abstr$ for some $l_a' \in L_a$, and (2) for every $l_a' \in L_a$ such that $((l_a,A_t),(l_a,A_t')) \in \trans_\abstr$, there is a child $v'$ of $v$ labelled with $(l_a',A_t')$.
\end{itemize}


A \emph{concrete counterexample tree} $\counterex_\belief$ for $(G_\belief,\LTLglobally p_k)$ is a finite tree with nodes labelled with states in $\states_\belief$ where:
\begin{itemize}
\item The root node is labelled with the initial state $s_\belief^\init$;
\item A node is labelled with a belief state which violates $p_k$ (that is, $s_\belief$ where $s_\belief \not\models p_k$) iff it is a leaf;
\item The tree branches according to all possible transition choices of the agent. Formally, if an internal node $v$ is labelled with $(l_a,B_t)$, then there exists a $B_t'$  such that: (1) $((l_a,B_t),(l_a',B_t')) \in \trans_\belief$ for some $l_a' \in L_a$, and (2) for every $l_a' \in L_a$ such that $((l_a,B_t),(l_a,B_t')) \in \trans_\belief$, there is a child $v'$ of $v$ labelled with $(l_a',B_t')$.
\end{itemize}

Due to the overapproximation of the belief-sets, not every counterexample in the abstract game corresponds to a winning strategy for the target in the original game.

An abstract counterexample $\counterex_\abstr$ in $(G_\abstr,\LTLglobally p_k)$ is \emph{concretizable} if there exists a concrete counterexample 
tree $\counterex_\belief$ in $(G_\belief,\LTLglobally p_k)$, which differs from $\counterex_\abstr$ only in the node labels, and each node labelled with $(l_a,A_t)$ in $\counterex_\abstr$ has label $(l_a, B_t)$ in $\counterex_\belief$ for which $B_t \subseteq \gamma(A_t)$.


\begin{figure}
\subfloat[Abstract counterexample tree\label{fig:simple-safety-counterex-abstr}]{
\begin{tikzpicture}[node distance=.9 cm,auto,>=latex',line join=bevel,transform shape,scale=.8]
\node at (0,0) (s0) {$(12,32)$};
\node  [below left of=s0,yshift=-.5cm,xshift=-.5cm] (s1) {$(11,\{Q_4,Q_5\})$};
\node  [below right of=s0,yshift=-.5cm,xshift=.5cm] (s2) {$(19,\{Q_4,Q_5\})$};

\draw [->] (s0) edge (s1.north);
\draw [->] (s0) edge (s2.north);
\end{tikzpicture}

}
\hfill
\subfloat[Concrete counterexample tree\label{fig:simple-safety-counterex-concr}]{
\begin{tikzpicture}[node distance=.9 cm,auto,>=latex',line join=bevel,transform shape,scale=.8]
\node at (0,0) (s0) {$(4,18)$};
\node  [below left of=s0,yshift=-.5cm,xshift=-.5cm] (s1) {$(3,\{17,23\})$};
\node  [below right of=s0,yshift=-.5cm,xshift=.5cm] (s2) {$(9,\{17,23\})$};

\draw [->] (s0) edge (s1.north);
\draw [->] (s0) edge (s2.north);
\end{tikzpicture}

}
\caption{Abstract and corresponding concrete counterexample trees for the surveillance game in Example~\ref{ex:simple-safety-counterex}.}
\label{fig:simple-safety-counterex}
\end{figure}

\begin{example}\label{ex:simple-safety-counterex}
Figure~\ref{fig:simple-safety-counterex-abstr} shows an abstract counterexample tree $\counterex_\abstr$ for the game $(\alpha_\part(G),\LTLglobally p_1)$, where $G$ is the surveillance game structure from Example~\ref{ex:simple-surveillance-game} and $\part$ is the abstraction partition from Example~\ref{ex:simple-abstr-game}. The counterexample corresponds to the choice of the target to move to one of the locations $31$ or $39$, which, for every possible move of the agent, results in an abstract state with abstract belief $B = \{Q_4,Q_5\}$ violating $p_1$.
A concrete counterexample tree $\counterex_\belief$ concretizing $\counterex_\abstr$ is shown in Figure~\ref{fig:simple-safety-counterex-concr}.
\qed
\end{example}
