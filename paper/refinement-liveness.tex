\subsubsection{Forward belief-set propagation}

To check if an abstract counterexample graph $\counterex_\abstr$ is concretizable, we construct a finite graph $\mathcal{D}$ whose nodes are labelled with elements of $\states_\belief$ and with nodes of $\counterex_\abstr$.
By construction we will ensure that if a node $d$ in $\mathcal D$ is labelled with $\langle(l_a,B_t),v \rangle$, where $(l_a,B_t)$ is a belief state, and $v$ is a node in $\counterex_\abstr$, then $v$ is labelled with $(l_a,A_t)$ in $\counterex_\abstr$, and $B_t \subseteq \gamma(A_t)$. 

Initially $\mathcal D$ contains a single node $d_0$ labelled with $\langle\{s_\belief^\init\},v_0\rangle$, where $v_0$ is initial node of $\counterex_\abstr$. Consider a node $d$ in $\mathcal D$ labelled with $\langle(l_a,B_t),v \rangle$. For every child $v'$ of $v$ in $\counterex_\abstr$, labelled with an abstract state $(l_a',A_t')$ we proceed as follows. We let ${B_t}' = \post(l_a,B_t) \cap \gamma(A_t')$. If there exists a node $d'$ in $\mathcal D$ labelled with $\langle (l_a',B_t'),v\rangle$, then we add an edge from $d$ to $d'$ in $\mathcal{D}$. Otherwise, we create such a node and add the edge. We continue until no more nodes and edges can be added to $\mathcal D$. The procedure is guaranteed to terminate, since both  the graph $\counterex_\belief$, and $\states_\belief$ are finite, and we add a node labelled $\langle s_\belief, v\rangle$ to $\mathcal D$ at most once.

If the graph $\mathcal D$ contains a reachable cycle (it suffices to consider simple cycles, i.e., without repeating intermediate nodes) $\rho = d_0,\ldots,d_n$ with $d_0 = d_n$ such that some $d_i$ is labelled with $(l_a,B_t)$ where $(l_a,B_t) \models p_k$, then we conclude that the abstract counterexample $\counterex_\abstr$ is not concretizable. If no such cycle exists, then $\mathcal D$ is a concrete counterexample graph for the belief game $(G_\belief,\LTLglobally\LTLfinally p_k)$. 

%\comment{
\begin{algorithm}[t] \small
\KwIn{surveillance game $(G,\LTLglobally\LTLfinally p_k)$, abstract counterexample graph $\counterex_\abstr$ with initial node $v_0$}
\KwOut{a path $\pi$ in a graph $\mathcal D$ or {\sc concretizable}}

\smallskip

graph $\mathcal D = (D,E)$ with nodes $D := \{d_0\}$ and edges $E := \emptyset$\;

annotate $d_0$ with $\langle s_\belief^\init, v_0\rangle$\; 

\SetKwRepeat{Do}{do}{while}

\Do{$\mathcal D \neq \mathcal D'$}{
$\mathcal D' := \mathcal D$\;
 \ForEach{node $d$ in $\mathcal D$ labelled with $\langle(l_a,B_t),v\rangle$}{
  \ForEach{child $v'$ of $v$ in $\counterex_\abstr$ labelled $(l_a',A_t')$}{
  $B_t' := \post(l_a,B_t)\cap\gamma(A_t')$\;
  \eIf{there is a node $d' \in D$ labelled with $\langle (l_a',B_t'),v'\rangle$}{add an edge $(d,d')$ to $E$}
  {add a node $d'$ labelled $\langle (l_a',B_t'),v'\rangle$ to $D$\;
  add an edge $(d,d')$ to $E$}
}
}
}
\leIf{there is a lasso path $\pi$ in $\mathcal D$ starting from $d_0$ such that some node in the cycle is annotated with $\langle s_\belief,v\rangle$ and $s_\belief\models p_k$\newline}
{\KwRet{$\pi$;}}
{\KwRet{{\sc concretizable}}}

\smallskip

\caption{Analysis of abstract counterexample graphs for games with liveness surveillance objectives.}
\label{algo:cex-analysis-liveness}
\end{algorithm}
%}

\begin{example}\label{ex:simple-liveness-unconcretizable}
The abstract counterexample graph in the game $(\alpha_\part(G),\LTLglobally \LTLfinally p_2)$ discussed in Example~\ref{ex:simple-liveness-counterex} is not conretizable, since for the path in the abstract graph depicted in Figure~\ref{fig:simple-liveness-counterex} there exists a corresponding path in the graph $\mathcal D$ with a node in the cycle labelled with a set in $G_\belief$ that satisfies $p_2$. More precisely, the cycle in the graph $\mathcal D$ contains a node labelled with $(19,\{10\})$. Intuitively, as the agent moves from the upper to the lower part of the grid along this path, upon not observing the target, it can infer from the sequence of observations that the only possible location of the target is $10$. Thus, this paths is winning for the agent.
\qed
\end{example}

\begin{theorem}
If Algorithm~\ref{algo:cex-analysis-liveness} returns a path $\pi$ in the graph $\mathcal D$ constructed for $\counterex_\abstr$, then $\counterex_\abstr$ is not concretizable, and the infinite run in $G_\belief$ corresponding to $\pi$ satisfies $\LTLglobally\LTLfinally p_k$, otherwise  $\counterex_\abstr$ is concretizable.
\end{theorem}

\subsubsection{Backward partition splitting}

Consider a path in the graph $\mathcal{D}$ of the form $\pi = d_0,\ldots, d_n,d_0',\ldots,d_m'$ where $d_n = d_m'$, and where for some $0 \leq i \leq m$ for the label $(l_a^i,B_t^i)$ it holds that $(l_a^i,B_t^i) \models p_k$. Let 
$\pi_\abstr = v_0,\ldots, v_n,v_0',\ldots,v_m'$ be the sequence of nodes in $\counterex_\abstr$ corresponding to the labels in $\pi$. By construction of $\mathcal D$, $\pi_\abstr$ is a path in $\counterex_\abstr$ and $v_n = v_m'$. We apply the refinement procedure from the previous section to the whole path $\pi_\abstr$, as well as to the to the path-prefix $v_0,\ldots, v_n$.

Let $\part$ and $\part'$ be two counterexample partitions such that $\part' \preceq \part$. Let $\counterex_\abstr$ be an abstract counterexample graph in $(\alpha_\part(G),\LTLglobally\LTLfinally p_k)$. We define $\gamma_{\part'}(\counterex_\abstr)$ to be the set of abstract counterexample graphs in $(\alpha_{\part'}(G),\LTLglobally\LTLfinally p_k)$ such that for every infinite path $\pi$ in $\counterex_\abstr'$ there exists an infinite path $\pi$ in $\counterex_\abstr$ such that for every node in $\pi'$ labelled with $(l_a,A_t')$ the corresponding node in $\counterex_\abstr$ is labelled with an abstract state $(l_a,A_t)$ such that $\gamma(A_t') \subseteq \gamma(A_t)$.

\begin{theorem}If $\part'$ is the partition 
%returned by Algorithm~\ref{algo:refinement-liveness} 
obtained by refining $\part$ with respect to an uncocretizable abstract counterexample $\counterex_\abstr$ in $(\alpha_\part(G),\LTLglobally\LTLfinally p_k)$, then $\gamma_{\part'}(\counterex_\abstr) = \emptyset$, and also $\gamma_{\part''}(\counterex_\abstr) = \emptyset$ for every partition $\part''$ with $\part'' \preceq \part'$.
\end{theorem}

\begin{example}\label{ex:simple-liveness-refinement}
We refine the abstraction partition $\part$ from Example~\ref{fig:simple-liveness-counterex} using the path identified there, in order to eliminate the abstract counterexample. For this, following the refinement algorithm, we first split the set $Q_1$ into sets $Q_1' = \{10\}$ and $Q_2' = Q_1 \setminus \{10\}$, and let $Q_3' = Q_2$. However, since from some locations in $Q_2'$ and in $Q_3'$ the target can reach locations in $Q_2'$ and $Q_3'$ that are not visible from the agent's position $19$, in order to eliminate the counterexample, we need to propagate the refinement backwards along the path and split $Q_2'$ and $Q_3'$ further. With that, we obtain an abstraction partition with $10$ sets, which is guaranteed to eliminate this abstract counterexample. In fact, in this example this abstraction turns out to be sufficiently precise to obtain a winning strategy for the agent.
\qed
\end{example}
