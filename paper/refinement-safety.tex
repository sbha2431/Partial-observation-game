\subsubsection{Forward belief-set propagation}
Given an abstract counterexample tree $\counterex_\abstr$ we label its nodes with states in $\states_\belief$ in a top-down manner as follows. 
The root node is labelled with $s_\belief^\init$. If a node $v$ is a node labelled with the belief set $(l_a,B_t) \in \states_\belief$, and  $v'$ is a child of $v$ in $\counterex_\abstr$ labelled with an abstract state $(l_a',A_t')$, then we label $v'$ with the belief set $(l_a',B_t')$, where 
$B_t' = \post(B_t) \cap \gamma(A_t')$. The counterexample analysis procedure based on this annotation is given in Algorithm~\ref{algo:cex-analysis-safety}.
If each of the leaf nodes of the tree is annotated with a belief set $(l_a,B_t)$ for which $(l_a,B_t) \not\models p_k$, then the new annotation gives us a concrete counterexample tree $\counterex_\belief$, which by construction concertizes $\counterex_\abstr$. Conversely, if there exists a leaf node labelled with $(l_a,B_t)$ such that $(l_a,B_t) \models p_k$, then we can conclude that the abstract counterexample tree $\counterex_\abstr$ is not concretizable and use the path from the root of the tree to this leaf node to refine the partition $\part$.

\begin{theorem}\todo{correctness of counterexample analysis}
\end{theorem}

\begin{figure}
\begin{center}
\begin{tikzpicture}[node distance=.9 cm,auto,>=latex',line join=bevel,transform shape,scale=.75]
\node at (0,0) (s0) {$v_0:(12,32)$};
\node  [below left of=s0,yshift=-.5cm,xshift=-2.2cm] (s1) {$v_1:(11,\{Q_1\})$};
\node  [below right of=s0,yshift=-.5cm,xshift=2.2cm] (s2) {$v_2:(19,\{Q_1\})$};
\node  [below of=s1,yshift=-.5cm] (s4) {$v_4:(12,A)$};
\node  [below of=s2,yshift=-.5cm] (s7) {$v_7:(18,A)$};
\node  [left of=s4,xshift=-1cm] (s3) {$v_3:(10,A)$};
\node  [right of=s4,xshift=1cm] (s5) {$v_5:(18,A)$};
\node  [left of=s7,xshift=-1cm] (s6) {$v_6:(12,A)$};
\node  [right of=s7,xshift=1cm] (s8) {$v_8:(26,A)$};
\draw [->] (s0) edge (s1.north);
\draw [->] (s0) edge (s2.north);
\draw [->] (s1) edge (s3.north);
\draw [->] (s1) edge (s4.north);
\draw [->] (s1) edge (s5.north);
\draw [->] (s2) edge (s6.north);
\draw [->] (s2) edge (s7.north);
\draw [->] (s2) edge (s8.north);

\end{tikzpicture}

\end{center}
\caption{Abstract counterexample in Example~\ref{ex:simple-safety-unconcretizable}. The leaf nodes are labelled with the abstract belief set $A = \{Q_1,Q_2\}$.}
\label{fig:simple-safety-counterex-1}
\end{figure}

\begin{example}\label{ex:simple-safety-unconcretizable}
Let again $G$ be the surveillance game structure from Example~\ref{ex:simple-surveillance-game}, and consider the surveillance game $(G,\LTLglobally p_5)$. 
Let $\part = \{Q_1,Q_2\}$ consist of the set $Q_1$, corresponding to the first two columns of the grid in Figure~\ref{simple-grid} and the set $Q_2$ containing the locations from the other three columns of the grid. Figure~\ref{fig:simple-safety-counterex-1} shows a counterexample tree $\counterex_\abstr$ in the abstract game $(\alpha_\part(G),\LTLglobally p_5)$. The analysis in Algorithm~\ref{algo:cex-analysis-safety} annotates node $v_1$ with the concrete belief set $\{31,39\}$, and the leaf node $v_3$ with the set $B = \{30,32,38\}$. Thus, this counterexample tree $\counterex_\abstr$ is determined to be spurious and the abstraction partition $\part$ should be refined.
\end{example}

\subsubsection{Backward partition splitting}
Consider a path $\pi_\abstr = v_0,\ldots, v_n$ in $\counterex_\abstr$ where $v_0$ is the root node and $v_n$ is a leaf. For each node $n_i$, let $(l_a^i,A_t^i) $ be the abstract state labelling $n_i$ in $\counterex_\abstr$, and let $(l_a^i,B_t^i)$ be the  belief set with which the node was labelled by the the counterexample analysis procedure. We consider the case when $(l_a^n,B_t^n) \models p_k$, that is, $|\{l_t \in B_t^n \mid \vis(l_a,l_t) = \false\}| \leq k$.
Note that since $\counterex_\abstr$ is a counterexample we have $(l_a^n,A_t^n) \not \models p_k$, and since $k>0$, this means $A_t \subseteq \mathcal{P}(\mathcal Q)$.


We now describe a procedure to compute a partition $\part'$ that refines the current partition $\part$ based on the path $\pi_\abstr$. Intuitively, we split the partition states that appear in $A_t^n$ with the goal to ensure that in the refined abstract game the corresponding abstract state satisfies the surveillance predicate $p_k$. To achieve this, however, we might have to also split partition elements appearing in abstract states on the path to $n_b$. The reason is that we have to ensure that earlier imprecisions on this path do not propagate and including more of the newly split partition states leading to the same violation of $p_k$.
Formally, if $A_t^n = (l_a^n,\{B_{n,1},\ldots,B_{n,m_n}\})$, then we split some of the partition states $B_{n,1},\ldots,B_{n,m_n}$ to obtain from $A_t^n$ a set $A_j' = \{B_{n,1}',\ldots,B_{n,m_n'}'\}$ such that
\[|\{l_t \in \gamma(C^n) \mid \vis(l_a^n,l_t) = \false\}| \leq k \text{, where}\] 
\[C^n = \{B_{n,i}' \in A_j' \mid B_{n,i}' \cap B_t^n \neq \emptyset\}.\]
This property intuitively means that if we consider the sets in $A'$ that have non-empty intersection with $B_t^n$, an abstract state composed of those partition sets will satisfy $p_k$. Since $(l_a^n,B_t^n)$ satisfies $p_k$, we can find a partition $\part^n \preceq \part$ that guarantees this property. What remains in order to eliminate this counterexample, is to ensure that only these partition states are reachable via the considered path, by propagating this splitting backwards, obtaining a sequence of partitions $\part \succeq \part^n \succeq \part^{n-1} \succeq \ldots \succeq \part^0$ refining $\part$. Given $\part^{j+1}$, we compute $Q^j$ as follows. For each $j$, we define a set $C^j \subseteq \mathcal{P}(L_t)$ (for $j=n$, the set $C^n$ was defined above). Suppose we have defined $C^{j+1}$ for some $j \geq 0$, and $A_t^j = (l_a^j,\{B_{j,1},\ldots,B_{j,m_j}\})$. We split some of the sets $B_{j,1},\ldots,B_{j,m_j}$ to obtain from $A_t^j$ a set $A_j' = \{B_{j,1}',\ldots,B_{j,m_j'}'\}$ where there exists $C^j \subseteq A_j'$ with
\[\gamma(C^j) = \gamma(A_t^j) \cap \{l_t \mid \post(l_a^j,l_t) \cap \gamma(A_t^{j+1}) \subseteq \gamma(C^{j+1})\}.\] 
Intuitively, this means that the new partition allow us to express precisely the set of states that do not lead to sets in $A_{j+1}'$ that we are trying to avoid. Again, the fact that an appropriate partition $\part$ can be computed follows from the choice of the leaf node $v_n$. The procedure for computing the partition $\part' = \part^0$ that refines $\part$ based on such a path $\pi_\abstr$ is formalized in Algorithm~\ref{}, and the theorem below states the progress property (eliminating the considered counterexample) which it guarantees.

\begin{example}
We continue with the unconcretizable abstract counterexample tree from Example~\ref{ex:simple-safety-unconcretizable}. We illustrate he refinement procedure for the path $v_0,v_1,v_3$. For node $v_3$, we split $Q_1$ and $Q_2$ using the set $B = \{30,32,38\}$, obtaining the sets $Q_1' = Q_1 \cap \{30,32,38\} = \{30\}$, $Q_2' = Q_1\setminus\{30\}$, $Q_3' = Q_2 \cap \{30,32,38\} = \{32,38\}$ and $Q_4' = Q_2 \setminus \{32,38\}$. We thus obtain a new partition $\part_{v_3} \preceq \part$. In order to propagate the refinement backwards (to ensure eliminating $\counterex_\abstr$) we compute the set of locations from which the target can move to a location in $Q_2'$ and $Q_4'$ that is not visible from location $10$. In this case, these are just the locations $32$ and $38$, which have already been separated from $Q_2$, so here backward propagation does not require further splitting.\qed
\end{example}

\begin{theorem}\todo{progress property of the refinement procedure: counterexample eliminated}
\end{theorem}