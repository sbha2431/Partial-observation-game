We now describe a procedure that determines whether an abstract counterexample for a safety surveillance objective is concretizable. This procedure essentially constructs the precise belief sets corresponding to the abstract moves of the target that constitute the abstract counterexample.

\subsubsection{Forward belief-set propagation}
Given an abstract counterexample tree $\counterex_\abstr$, we annotate its nodes with states in $\states_\belief$ in a top-down manner as follows. 
The root node is labelled with $s_\belief^\init$. 
If $v$ is a node annotated with the belief set $(l_a,B_t) \in \states_\belief$, and  $v'$ is a child of $v$ in $\counterex_\abstr$ labelled with an abstract state $(l_a',A_t')$, then we annotate $v'$ with the belief set $(l_a',B_t')$, where 
$B_t' = \post(l_a,B_t) \cap \gamma(A_t')$. The counterexample analysis procedure based on this annotation is given in Algorithm~\ref{algo:cex-analysis-safety}.
If each of the leaf nodes of the tree is annotated with a belief set $(l_a,B_t)$ for which $(l_a,B_t) \not\models p_k$, then the new annotation gives us a concrete counterexample tree $\counterex_\belief$, which by construction concertizes $\counterex_\abstr$. Conversely, if there exists a leaf node annotated with $(l_a,B_t)$ such that $(l_a,B_t) \models p_k$, then we can conclude that the abstract counterexample tree $\counterex_\abstr$ is not concretizable and use the path from the root of the tree to this leaf node to refine the partition $\part$.
%\Suda{We can move this algorithm to the appendix right? The process is described in the previous paragraph anyway.}

%\comment{
\begin{algorithm}[b]
\vspace{-.4cm}
\small
\KwIn{surveillance game $(G,\LTLglobally p_k)$,\newline abstract counterexample tree $\counterex_\abstr$}
\KwOut{a path $\pi$ in $\counterex_\abstr$ or {\sc concretizable}}

%\smallskip

\While{there is a node $v$ in $\counterex_\abstr$ whose children\newline are not annotated with states in $\states_\belief$}{
 let $(l_a,B_t)$ be the state with which $v$ is annotated\;
 \ForEach{child $v'$ of $v$ labelled with $(l_a',A_t')$}{
 annotate $v'$ with $(l_a',\post(l_a,B_t)\cap\gamma(A_t'))$\;
}
}%
\leIf{there is a path $\pi$ in $\counterex_\abstr$ from the root to a leaf annotated with a sate $s_\belief$ where $s_\belief\models p_k$\newline}
{\KwRet{$\pi$;}}
{\KwRet{{\sc concretizable}}}

%\smallskip


\caption{Analysis of abstract counterexample trees for games with safety surveillance objectives.}
\label{algo:cex-analysis-safety}
\end{algorithm}
%}

\begin{theorem}
If Algorithm~\ref{algo:cex-analysis-safety} returns a path $\pi_\abstr$ in $\counterex_\abstr$, then $\counterex_\abstr$ is not concretizable, and $\pi_\abstr$ is a non-concretizable path, otherwise  $\counterex_\abstr$ is concretizable.
\end{theorem}

\begin{figure}
\begin{center}
\begin{tikzpicture}[node distance=.9 cm,auto,>=latex',line join=bevel,transform shape,scale=.75]
\node at (0,0) (s0) {$v_0:(12,32)$};
\node  [below left of=s0,yshift=-.5cm,xshift=-2.2cm] (s1) {$v_1:(11,\{Q_1\})$};
\node  [below right of=s0,yshift=-.5cm,xshift=2.2cm] (s2) {$v_2:(19,\{Q_1\})$};
\node  [below of=s1,yshift=-.5cm] (s4) {$v_4:(12,A)$};
\node  [below of=s2,yshift=-.5cm] (s7) {$v_7:(18,A)$};
\node  [left of=s4,xshift=-1cm] (s3) {$v_3:(10,A)$};
\node  [right of=s4,xshift=1cm] (s5) {$v_5:(18,A)$};
\node  [left of=s7,xshift=-1cm] (s6) {$v_6:(12,A)$};
\node  [right of=s7,xshift=1cm] (s8) {$v_8:(26,A)$};
\draw [->] (s0) edge (s1.north);
\draw [->] (s0) edge (s2.north);
\draw [->] (s1) edge (s3.north);
\draw [->] (s1) edge (s4.north);
\draw [->] (s1) edge (s5.north);
\draw [->] (s2) edge (s6.north);
\draw [->] (s2) edge (s7.north);
\draw [->] (s2) edge (s8.north);

\end{tikzpicture}

\end{center}
\vspace{-.2cm}
\caption{Abstract counterexample in Example~\ref{ex:simple-safety-unconcretizable}. The leaf nodes are labelled with the abstract belief set $A = \{Q_1,Q_2\}$.}
\label{fig:simple-safety-counterex-1}
\vspace{-.5cm}
\end{figure}

\begin{example}\label{ex:simple-safety-unconcretizable}
Let $G$ be the surveillance game structure from Example~\ref{ex:simple-surveillance-game}, and consider the surveillance game $(G,\LTLglobally p_5)$. 
Let $\part = \{Q_1,Q_2\}$ consist of the set $Q_1$, corresponding to the first two columns of the grid in Figure~\ref{simple-grid} and the set $Q_2$ containing the locations from the other three columns of the grid. Figure~\ref{fig:simple-safety-counterex-1} shows a counterexample tree $\counterex_\abstr$ in the abstract game $(\alpha_\part(G),\LTLglobally p_5)$. The analysis in Algorithm~\ref{algo:cex-analysis-safety} annotates node $v_1$ with the concrete belief set $\{17,23\}$, and the leaf node $v_3$ with the set $B = \{16,18,22\}$. Thus, this counterexample tree $\counterex_\abstr$ is determined to be unconcretizable and the partition $\part$ should be refined.\qed
\end{example}

When the analysis procedure determines that an abstract counterexample tree is unconcretizable, it returns a path in the tree that corresponds to a sequence of moves ensuring that the size of the belief-set does not actually exceed the threshold, given that the target behaves in a way consistent with the abstract counterexample.  Based on this path, we can then refine the abstraction in order to precisely capture this information and thus eliminate this abstract counterexample.


\subsubsection{Backward partition splitting}
Let $\pi_\abstr = v_0,\ldots, v_n$ be a path in $\counterex_\abstr$ where $v_0$ is the root node and $v_n$ is a leaf. For each node $v_i$, let $(l_a^i,A_t^i) $ be the abstract state labelling $v_i$ in $\counterex_\abstr$, and let $(l_a^i,B_t^i)$ be the  belief set with which the node was annotated by the counterexample analysis procedure. We consider the case when $(l_a^n,B_t^n) \models p_k$, that is, $|\{l_t \in B_t^n \mid \vis(l_a,l_t) = \false\}| \leq k$.
Note that since $\counterex_\abstr$ is a counterexample we have $(l_a^n,A_t^n) \not \models p_k$, and since $k>0$, this means $A_t \in \mathcal{P}(\part)$.


We now describe a procedure to compute a partition $\part'$ that refines the current partition $\part$ based on the path $\pi_\abstr$. Intuitively, we split the sets that appear in $A_t^n$ in order to ensure that in the refined abstract game the corresponding abstract state satisfies the surveillance predicate $p_k$. We may have to also split sets appearing in abstract states on the path to $v_n$, as we have to ensure that earlier imprecisions on this path do not propagate, thus including more than the desired newly split sets, and leading to the same violation of $p_k$.

Formally, if $A_t^n = (l_a^n,\{B_{n,1},\ldots,B_{n,m_n}\})$, then we split some of the sets $B_{n,1},\ldots,B_{n,m_n}$ to obtain from $A_t^n$ a set $A'_n = \{B_{n,1}',\ldots,B_{n,m_n'}'\}$ such that

$\qquad|\{l_t \in \gamma(C^n) \mid \vis(l_a^n,l_t) = \false\}| \leq k \text{, where}$

$\qquad\qquad C^n = \{B_{n,i}' \in A'_n \mid B_{n,i}' \cap B_t^n \neq \emptyset\}.$

This property intuitively means that if we consider the sets in $A'_n$ that have non-empty intersection with $B_t^n$, an abstract state composed of those sets will satisfy $p_k$. Since $(l_a^n,B_t^n)$ satisfies $p_k$, we can find a partition $\part^n \preceq \part$ that guarantees this property, as shown in Algorithm~\ref{algo:refinement-safety}.
 
What remains, in order to eliminate this counterexample, is to ensure that only these sets are reachable via the considered path, by propagating this split backwards, obtaining a sequence of partitions $\part \succeq \part^n \succeq \ldots \succeq \part^0$ refining $\part$. 

Given $\part^{j+1}$, we compute $\part^j$ as follows. For each $j$, we define a set $C^j \subseteq \mathcal{P}(L_t)$ (for $j=n$, the set $C^n$ was defined above). Suppose we have defined $C^{j+1}$ for some $j \geq 0$, and $A_t^j = (l_a^j,\{B_{j,1},\ldots,B_{j,m_j}\})$. We split some of the sets $B_{j,1},\ldots,B_{j,m_j}$ to obtain from $A_t^j$ a set $A'_j = \{B_{j,1}',\ldots,B_{j,m_j'}'\}$ where there exists $C^j \subseteq A'_j$ with
\[\gamma(C^j) = \{l_t \in \gamma(A_t^j) \mid \post(l_a^j,\{l_t\}) \cap \gamma(A_t^{j+1}) \subseteq \gamma(C^{j+1})\}.\]
This means that using the new partition we can express precisely the set of states that do not lead to sets in $A_{j+1}'$ that we are trying to avoid. 
The fact that an appropriate partition $\part$ can be computed, follows from the choice of the leaf node $v_n$. 
The procedure for computing the partition $\part' = \part^0$ that refines $\part$ based on  $\pi_\abstr$ is formalized in Algorithm~\ref{algo:refinement-safety}.
\begin{example}
We continue with the unconcretizable abstract counterexample tree from Example~\ref{ex:simple-safety-unconcretizable}. We illustrate the refinement procedure for the path $v_0,v_1,v_3$. For node $v_3$, we split $Q_1$ and $Q_2$ using the set $B = \{16,18,22\}$, obtaining the sets $Q_1' = Q_1 \cap \{16,18,22\} = \{16\}$, $Q_2' = Q_1\setminus\{16\}$, $Q_3' = Q_2 \cap \{16,18,22\} = \{18,22\}$ and $Q_4' = Q_2 \setminus \{18,22\}$. We thus obtain a new partition $\part_{v_3} \preceq \part$. In order to propagate the refinement backwards (to ensure eliminating $\counterex_\abstr$) we compute the set of locations from which the target can move to a location in $Q_2'$ or $Q_4'$ that is not visible from location $2$. In this case, these are just the locations $18$ and $22$, which have already been separated from $Q_2$, so here backward propagation does not require further splitting.\qed
\end{example}

%\comment{
\begin{algorithm}[b]
\vspace{-.4cm}
\small
\KwIn{surveillance game $(G,\LTLglobally p_k)$, abstraction partition $\part$,\newline unconcretizable path $\pi = v_0,\ldots,v_n$ in $\counterex_\abstr$}
\KwOut{an abstraction partition $\part'$ such that $\part' \preceq \part$}

%\smallskip

let $(l_a^j,A_t^j)$ be the label of $v_j$, and $(l_a^j,B_t^j)$ its annotation;
\begin{flalign*}
A  :=&  \{Q \cap B_t^n\mid Q \in A_t^n , Q \cap B_t^n \neq \emptyset \}\cup &\\
& \{Q \setminus B_t^n\mid Q \in A_t^n , Q \setminus B_t^n \neq \emptyset \};
\end{flalign*}

$\part' := (\part \setminus A_t^n )  \cup A$;
$C := \{ Q\in A \mid Q \cap B_t^n \neq \emptyset\}$

\For{$j = n-1,\ldots,0$}{
\lIf{$A_t^j \in L_t$}{{\bf break}}
$B :=  \{l_t \in \gamma(A_t^j) \mid \post(l_a,\{l_t\}) \subseteq \gamma(C)\}$\;
\noindent
\begin{flalign*}
A  := &\{Q \cap B\mid Q \in A_t^j , Q \cap B \neq \emptyset \}\cup &\\
& \{Q \setminus B\mid Q \in A_t^j , Q \setminus B \neq \emptyset \};
\end{flalign*}


$\part' := (\part' \setminus A_t^j )  \cup A$;
$C := \{ Q\in A \mid Q \cap B \neq \emptyset\}$
} 
\KwRet{$\part'$}

%\smallskip

\caption{Abstraction partition refinement given an unconcretizable path in an  abstract counterexample tree.}
\label{algo:refinement-safety}

\end{algorithm}
%}

Let $\part$ and $\part'$ be two counterexample partitions such that $\part' \preceq \part$. Let $\counterex_\abstr$ be an abstract counterexample  tree in $(\alpha_\part(G),\LTLglobally p_k)$. We define $\gamma_{\part'}(\counterex_\abstr)$ to be the set of abstract counterexample trees in $(\alpha_{\part'}(G),\LTLglobally p_k)$ such that $\counterex'_\abstr \in \gamma_{\part'}(\counterex_\abstr)$ iff $\counterex'_\abstr$ differs from $\counterex_\abstr$ only in the node labels and for every node in $\counterex_\abstr$ labelled with $(l_a,A_t)$, the corresponding node in $\counterex'_\abstr$ is labelled with an abstract state $(l_a,A_t')$ such that $\gamma(A_t') \subseteq \gamma(A_t)$.
 
The theorem below states the progress property (eliminating the considered counterexample) of Algorithm~\ref{algo:refinement-safety}.

\begin{theorem}If $\part'$ is the partition returned by Algorithm~\ref{algo:refinement-safety} for an unconcretizable abstract counterexample $\counterex_\abstr$ in $(\alpha_\part(G),\LTLglobally p_k)$, then $\gamma_{\part'}(\counterex_\abstr) = \emptyset$, and also $\gamma_{\part''}(\counterex_\abstr) = \emptyset$ for every partition $\part''$ where $\part'' \preceq \part'$.
\end{theorem}

\begin{example}\label{ex:simple-safety-realizability}
In the surveillance game $(G,\LTLglobally p_5)$, where $G$ is the surveillance game structure from Example~\ref{ex:simple-surveillance-game}, the agent has a winning strategy. After $6$ iterations of the refinement loop we arrive at an abstract game $(\alpha_{\part^*}(G),\LTLglobally p_5)$, where the partition $\part^*$ consists of $11$ automatically computed sets (as opposed to the $22$ locations reachable by the target in $G$), which in terms of the belief-set construction means $2^{11}$ versus $2^{22}$ possible belief sets in the respective games.

In the game $(G,\LTLglobally p_2)$, on the other hand, the agent does not have a winning strategy, and our algorithm establishes this after one refinement, after which, using a partition of size $4$,  it finds a concretizable abstract counterexample.
\qed
\end{example}