\subsection{Surveillance Game Structures}
We define a \emph{surveillance game structure} to be  a tuple $G  = (\states,s^\init,T,\vis)$, where:
\begin{itemize}
\item $\states = L_a \times L_t$ is the set of states with $L_a$ the set of locations of the agent and $L_t$ the locations of the target;
\item $s^\init = (l_a^\init,l_t^\init)$ is the initial state;
\item $\trans \subseteq \states \times \states$ is the transition relation describing the possible moves of the agent and the target;
\item $\vis : \states \to \bools$ is a function that maps a state $(l_a,l_t)$ to $\true$ if and only if\emph{ $l_t$ is in the line of sight of $l_a$}.
\end{itemize}

\begin{example}
\todo{grid plus small part of corresponding game structure}
\end{example}
\subsection{Belief-Set Game Structures}

For a surveillance game structure $G  = (\states,s^\init,T,\vis)$ we define the corresponding \emph{belief game structure} $G_\belief  = (\states_\belief,s^\init_\belief,T_\belief)$ with the following components
\begin{itemize}
\item $\states_\belief = L_a \times \mathcal P(L_t)$ is the set of states with $L_a$ the set of locations of the agent, and $\mathcal P(L_t)$ the set of belief sets describing information about the location of the target;
\item $s^\init_\belief = (l_a^\init,\{l_t^\init\})$ is the initial state;
\item $\trans_\belief \subseteq \states_\belief \times \states_\belief$ is the transition relation such that $((l_a, B_t),(l_a', B_t')) \in T_\belief$ if and only if one of the following three conditions is satisfied:
\begin{itemize}
\item[(1)] $B_t' = \{l_t'\}$ for some $l_t'$ such that $\vis(l_a,l_t') = \true$ and
there exists $l_t \in B_t$ with $((l_a,l_t),(l_a',l_t')) \in T$;
\item[(2)] there exists $l_t \in B_t$ such that $\vis(l_a,l_t) = \true$ and 
$\begin{array}{lll}
B_t' = \{l_t' & \mid & \vis(l_a,l_t') = \false \text{ and } \\
&& \text{and }  ((l_a,l_t),(l_a',l_t')) \in T\}.
\end{array}
$
\item[(3)] $\begin{array}{lll}
B_t' = \{l_t' & \mid & \vis(l_a,l_t') = \false \text{ and } \\
&& \exists l_t \in B_t: \vis(l_a,l_t') = \false \\
&& \text{and }  ((l_a,l_t),(l_a',l_t')) \in T\}.
\end{array}
$
\end{itemize}
The first condition captures the successor locations of the target that can be observed from the agent's current location $l_a$. Conditions (2) and (3) correspond to belief sets consisting of all possible successor locations of the target not visible from $l_a$. In (2) those are successors of a single possible current position $l_t$ of the target that is visible from $l_a$, while in (3) the belief consist of  successors of all positions in $B_t$ not visible from $l_a$.
\end{itemize}

\noindent{\bf Remark} \todo{explain sequence of agent's observation and action}
\begin{example}
\todo{small part of corresponding belief-set game structure}
\end{example}

\todo{run, strategy, outcome of strategy, ...}

\subsection{Temporal Quantitative Surveillance Objectives}

We consider a set of \emph{surveillance predicates} $\SP = \{p_k \mid k \in \nats\}$, where for $k \in \nats$ we say that a state $(l_a,B_t)$ in the belief game structure satisfies $p_k$ (denoted $(l_a,B_t) \models p_k$) iff 
$|\{l_t \in B_t \mid \vis(l_a,l_t)  = \false \}| \leq k$. Intuitively, $p_k$ is satisfied by the states in the belief game structure where the size of the belief set does not exceed the threshold $k \in \nats$.

We consider surveillance objectives expressed by formulas of linear temporal logic (LTL) over surveillance predicates. Since we are only interested in surveillance predicates that upper-bound the size of belief sets, we consider LTL formulas in negation normal form, in which disallow the occurrence of negation in front of surveillance predicates. Our LTL surveillance formulas  are generated by the grammar
\[\varphi := p \mid \true \mid \false \mid \varphi \wedge \varphi \mid \varphi \vee \varphi \mid \LTLnext  \varphi  \mid \varphi \LTLuntil \varphi \mid \varphi \LTLrelease \varphi,\]

where $p \in \SP$ is a surveillance predicate, $\LTLnext$ is the \emph{next} operator, $\LTLuntil$ is the \emph{until} operator, and $\LTLrelease$ is the \emph{release} operator. We also define the derived operators 
\emph{finally} $\LTLfinally \varphi = \true \LTLuntil \varphi$ and 
\emph{globally} $\LTLglobally \varphi = \false \LTLrelease \varphi$.

Of special interest will be surveillance formulas of the form $\LTLglobally p_k$, termed \emph{safety surveillance objective}, and $\LTLglobally\LTLfinally p_k$, called \emph{liveness surveillance objective}.
Intuitively, the safety surveillance formula $\LTLglobally p_k$ is satisfied if at each point in time the size of the belief set does not exceed $k$. The liveness surveillance objective $\LTLglobally\LTLfinally p_k$, on the other hand, requires that infinitely often this size is below or equal to $k$.

\todo{formal definition of LTL semantics (reference and in appendix), here examples}

\begin{example}
\todo{safety surveillance objective; liveness surveillance objective}
\end{example}


\subsection{Incorporating Functional Objectives}
We can easily integrate functional LTL objectives not related to surveillance by considering in addition to $\SP$ a set $\AP$ of predicates interpreted over states of $G$. In order to define the semantics of $p \in \AP$ over states of $G_\belief$, we restrict ourselves to predicates referring only to the position of the agent. Formally, we require that for $p \in \AP$, and states $(l_a,l_t')$ and $(l_a,l_t'')$, it holds that  $(l_a,l_t') \models p$ iff $(l_a,l_t'') \models p$. 

\begin{example}
\todo{safety surveillance objective + liveness objective, liveness surveillance objective + liveness objective}
\end{example}

\subsection{Surveillance Synthesis Problem}
A \emph{surveillance game} is a pair $(G,\varphi)$, where $G$ is a surveillance game structure and $\varphi$ is a surveillance objective.

\todo{problem statement}

