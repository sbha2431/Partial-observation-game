\subsection{Preliminaries}

\subsection{Surveillance Games}
Game structure $G  = (\states,s^\init,T,\vis)$, where
\begin{itemize}
\item $\states = L_a \times L_t$ is the set of states with $L_a$ the set of locations of the agent and $L_t$ the locations of the target,
\item $s^\init = (l_a^\init,l_t^\init)$ is the initial state,
\item $\trans \subseteq \states \times \states$ is the transition relation describing the possible moves of the agent and the target
\item $\vis : \states \to \bools$ is a function that maps a state $(l_a,l_t)$ to $\true$ if and only if $l_t$ is in the line of sight of $l_a$.
\end{itemize}

\subsection{Belief-Set Games}

Game structure $G_\belief  = (\states_\belief,s^\init_\belief,T_\belief)$, where
\begin{itemize}
\item $\states_\belief = L_a \times \mathcal P(L_t)$ is the set of states with $L_a$ the set of locations of the agent and $\mathcal P(L_t)$ the set of belief sets describing information about the location of the target
\item $s^\init_\belief = (l_a^\init,\{l_t^\init\})$ is the initial state,
\item $\trans_\belief \subseteq \states_\belief \times \states_\belief$ is the transition relation such that $((l_a, B_t),(l_a', B_t')) \in T_\belief$ if and only if some of the following two conditions is satisfied
\begin{itemize}
\item $B_t' = \{l_t'\}$ for some $l_t'$ such that $\vis(l_a',l_t') = \true$ and
there exists $l_t \in B_t$ with $((l_a,l_t),(l_a',l_t')) \in T$;
\item $\begin{array}{lll}
B_t' = \{l_t' & \mid & \vis(l_a',l_t') = \false \text{ and }\\
&&\exists l_t \in B_t.\; ((l_a,l_t),(l_a',l_t')) \in T\}
\end{array}
$. 

The first condition captures the successor locations of the target that can be observed from the agent's successor location, while the second  corresponds to the belief set that consists of all possible successor locations of the target that are not visible from the corresponding successor location of the agent.
\end{itemize}
\Rayna{In the above definition of the transition relation, the successor belief set depends on the successor position of the agent. Note, that the actual transitions of the target according to $T$ do not depend on the successor position of the agent, only the partitioning into visible/invisible locations does. However, this is still a problem if we want the successor $B_t'$ to depend only on $(l_a,l_t)$. This can be resolved by introducing intermediate states and strict turn-taking.}

\end{itemize}

Given a threshold value $k \in \nats_{>0}$, the surveillance objective that the size of the belief set does not exceed $k$ can be formulated as a safety objective with set of error states $\states_\belief^{>k} = \{(l_a, B_t) \in S_\belief \mid |B_t| > k\}$.


\begin{example}[Belief-update with intermediate states]
Consider agent locations $l_a,l_a^1,l_a^2$ and target locations 
$l_t^1,l_t^2,l_t^3,l_t^4,l_t^5,l_t^6$ with the following visibility:
\begin{itemize}
\item $\vis(l_a,l_t) =\false$ for all $l_t \in \{l_t^,\ldots,l_t^6\}$,
\item $\vis(l_a^1,l_t^3) =\true$ and $\vis(l_a,l_t^i) =\false$ if $i \neq 3$,
\item $\vis(l_a^2,l_t^4) =true$, $\vis(l_a^2,l_t^5) =\true$ and\\ $\vis(l_a,l_t^i) =\false$ if $i \not \in \{4,5\}$.
\end{itemize}
In the game the following transitions are possible:
\begin{itemize}
\item the agent can go from $l_a$ to $l_a^1$ and $l_a^2$,
\item the environment can go from $l_t^1$ to $l_t^3$ and to $l_t^4$,
\item the environment can go from $l_t^2$ to $l_t^5$ and to $l_t^6$.
\end{itemize}


We will consider states consisting of the location of the agent, the belief set (consisting of the possible locations of the target), and an extra bit controlled by the environment.

We have the following  possible transitions in the corresponding belief-set game:
\begin{itemize}
\item from $(l_a,\{l_t^1,l_t^2\},0)$ to $(l_a^1,\{l_t^3,l_t^4,l_t^5,l_t^6\},1)$,
\item from $(l_a,\{l_t^1,l_t^2\},0)$ to $(l_a^2,\{l_t^3,l_t^4,l_t^5,l_t^6\},1)$,
\item from $(l_a^1,\{l_t^3,l_t^4,l_t^5,l_t^6\},1)$ to $(l_a^1,\{l_t^3\},0)$,
\item from $(l_a^1,\{l_t^3,l_t^4,l_t^5,l_t^6\},1)$ to $(l_a^1,\{l_t^4,l_t^5,l_t^6\},0)$,
\item from $(l_a^2,\{l_t^3,l_t^4,l_t^5,l_t^6\},1)$ to $(l_a^2,\{l_t^3,l_t^6\},0)$,
\item from $(l_a^2,\{l_t^3,l_t^4,l_t^5,l_t^6\},1)$ to $(l_a^2,\{l_a^4\},0)$,
\item from $(l_a^2,\{l_t^3,l_t^4,l_t^5,l_t^6\},1)$ to $(l_a^2,\{l_a^5\},0)$.

Thus, in states of the from $(...,...,0)$ normal transitions take place, \emph{without} splitting the belief state according to the visibility from the new location of the agent, but setting the auxiliary bit to $1$. Transitions emanating from states of the form $(...,...,1)$ leave the location of the agent and the target unchanged, but the environment updates the belief sets according to the current visibility of the agent and sets the auxiliary bit to $1$, so that from the successor state normal transitions can be taken.




\end{itemize}

\end{example}