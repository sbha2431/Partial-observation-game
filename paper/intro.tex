Performing surveillance on an adversarial target, by its very nature, is a partial information problem. The agent may not always know the position of the target. However, surveillance in conjunction with a mission specification can be crucial in applications such as defense where it is important to keep track of (potentially hostile) targets whilst trying to satisfy a particular objective. 

Since we are dealing with an adversarial target, a natural setting for formulating the problem is a two-player game. There are several flavours of partial information games that have been studied in the literature \cite{Chatterjee2013}, and in this paper we focus on turn-based one-sided partial-observation deterministic games on which we perform reactive control synthesis. It is one-sided as we allow the adversary full information on the location of the agent even if it is not in sight. 

Related work in dealing with surveillance type objectives are pursuit-evasion games. There are several methods in formulating the problem such as enforcing eventual detection (at which point the game ends) \cite{Chen2010} or not allowing the target to move more than a certain distance away (\cite{keylist}). Partial information version of these games have also been studied in \cite{Antoniades2003,keylist} where it is shown that there is no existence theory for optimal solutions. However, these approaches treat the surveillance requirement as a search problem - once the target is detected the game is over. Work done by \cite{Vidal2002} and \cite{Kim2001} also include map building during pursuit but again with the sole purpose of target detection. Additionally, we do not enforce the requirement that the target(s) be detected. It is sufficient if we are able to bound our belief of the location below a user specified threshold for an infinite execution which allows for more richer, and more complex, behaviour than standard pursuit-evasion. So while there is work in pursuit-evasion games with partial information \cite{Chen2010} and even unknown environments \cite{Vidal2002}, these do not deal with additional mission specifications and also do not allow for the target to 'escape' and be recaptured as will be necessary in our multi-objective setting.

Our aim is to then synthesize a reactive controller that satisfies both the LTL specification as well as the surveillance objective. While it has been shown that for a general LTL specification, the synthesis problem is doubly exponential in the length of the formula \cite{Pnueli1989}, the work in \cite{Piterman2006} lays out a class of formulae called GR(1) that is $\mathcal{O}(N^3)$. This framework has been used extensively in robotic planning, for example in \cite{wong2012,Kress2007} and we do so here as well. We explicitly encode the surveillance requirement into the GR(1) formula to allow us to exploit the fast nature of GR(1) synthesis to solve a pursuit-evasion game. 

However, we still have the issue of partial observability in our setting. The controller will need to choose actions even when the state of the adversary is not known. The standard approach to deal with the partial observability is by using a \emph{belief set construction} to reduce the problem to a full observability game \cite{Bertoli2006}. However, the number of belief states will be exponential in the number of states \cite{Rintanen2004} as the belief set construction takes a powerset of the number of states. In general, this scales poorly and is not usable in most practical situations. \todo{literature on other partial information reduction heuristics}. In this paper, to deal with this problem, we introduce \emph{abstract belief set construction}. This is an underapproximation of the true belief space and hence, if a controller is found, then we know a controller will exist in the fully refined belief space. If a controller is not found, we use counterexample guided abstract refinement (CEGAR) to split a belief set and the process is repeated. While CEGAR has been extensively used on abstract models for GR(1) reactive synthesis \cite{Alur2015,keylist}, to our knowledge it has not been used on belief state refinement in reducing a partial information game to a full information game. 

The focus of this paper is to solve a modified pursuit-evasion game with additional LTL objectives in a partial information setting using reactive control synthesis. We propose a novel encoding of the surveillance requirement into the GR(1) specification to allow for assume guarantee control synthesis as well as a CEGAR approach in belief set construction in solving the partial information game. Our contributions in this paper are as follows:
\begin{itemize}
\item We encode the surveillance task as a safety specification which forces the agent to more closely follow the adversary in order to ensure the uncertainty (size of the belief set) on the location of adversary does not grow above the constraint.
\item We also encode surveillance task as a liveness objective. This allows for the agent to be more relaxed in monitoring the location of the agent if it can ensure that it can see it again sometime in the future.
\item We analyse the qualitatively different behaviour produced based on the specification type which allows the user to tailor the specification based on the requirements of the mission.
\item Avoiding the state space blow up by abstract belief set construction and using counter example guided belief refinement for both the safety and liveness specification cases.
\end{itemize}

The rest of the paper is structured as follows. In section II we provide definitions and notations for partial-observation games and the encoding of the pursuit-evasion requirement as safety and liveness objectives in LTL. In section III, we present our belief set abstraction in reducing the partial information game to a full information and also detail the CEGAR process for the different types of objectives. In section IV we provide experiments on gridworlds along with a simulation in ROS using our proposed algorithm, and we conclude and provide future direction in section V. 
