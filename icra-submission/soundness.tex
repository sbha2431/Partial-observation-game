In the construction of $G_\abstr$, we overapproximate the belief of the agent at each step. Since we consider surveillance predicates that impose upper bounds on the size of the belief, this approximation gives more power to the target (and, dually, less power to the agent) and guarantees that the abstraction is \emph{sound}, meaning that an abstract winning strategy for the agent corresponds to a strategy for the agent that achieves the surveillance objective in the concrete game. 
%This is stated in the following theorem.

\begin{theorem}
Let $G$ be a surveillance game structure, $\part = \{Q_i\}_{i=1}^n$ be an abstraction partition, and $G_\abstr = \alpha_\part(G)$. For every surveillance objective $\varphi$, if there exists a wining strategy for the agent in the game $(\alpha_\part(G),\varphi)$, then the agent has a  winning strategy in the surveillance game $(G,\varphi)$.
\end{theorem}
